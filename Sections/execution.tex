Die Durchführung des Labors gliederte sich in drei Hauptteile:
\begin{enumerate}
    \item Aufnahme der maximalen freiwilligen Kontraktion (MVC)
    \item Messung der relativen Muskelaktivität bei unterschiedlichen Belastungen
    \item Messung der Muskelerholung und Ermüdung
\end{enumerate}

Alle Messungen wurden unter identischen Systemeinstellungen und Elektrodenpositionen durchgeführt,
um die Vergleichbarkeit der Ergebnisse sicherzustellen. 
Die Elektroden und das Messsystem wurden entsprechend den vorbereitenden Arbeiten (Abschnitte~2.2 und~2.5) aufgebaut.

\subsection{Maximale freiwillige Kontraktion (MVC)}

Alle drei Gruppenmitglieder führten dieses Experiment durch, um ihre persönliche maximale Muskelaktivität zu bestimmen. 
Die Teilnehmer nahmen jeweils eine Sitzposition ein, bei der der Oberarm vertikal gehalten und der Unterarm um 90° gebeugt war, 
während die Hand die untere Tischkante umgriff. Die anderen Gruppenmitglieder stabilisierten den Tisch, 
sodass dieser während der isometrischen Kontraktion nicht bewegt wurde und der aktive Teilnehmer seine maximale Kraft aufbringen konnte.

Jeder Teilnehmer führte drei Durchläufe mit jeweils fünf Sekunden maximaler Anspannung durch, zwischen denen mindestens 60 Sekunden Pause lagen.
Die Rohdaten wurden in separaten CSV-Dateien gespeichert und dienten als Basis für die Berechnung der MVC, die als Referenz für die 
relative Muskelaktivität sowie für die Analyse der Muskelermüdung verwendet wurde.

Dieser Aufbau erlaubt dem Muskel, sich maximal zu kontrahieren, da der Unterarm fixiert ist und der Tisch als stabiler Widerstand dient.
Die isometrische Kontraktion stellt sicher, dass sich die Muskellänge nicht verändert, wodurch die maximale Muskelspannung erfasst wird.

\subsection{Relative Muskelaktivität}

Dieses Experiment wurde von einem Gruppenmitglied durchgeführt, um die Muskelaktivität bei unterschiedlichen Belastungen zu messen.
Die Sitzposition und Armhaltung entsprachen der MVC-Messung. Der Teilnehmer hielt nacheinander drei verschiedene Gewichte für jeweils 
zehn Sekunden, wobei zwischen den Durchgängen mindestens 40 Sekunden Pause eingelegt wurden.

Die EMG-Daten wurden für jeden Durchgang separat aufgezeichnet, um die relative Muskelaktivität als Prozentsatz der zuvor bestimmten MVC 
berechnen zu können. Die verwendeten Gewichte betrugen:
\begin{itemize}
    \item \textbf{Gewicht 1 (ca.\ 25\,\% MVC):} 5{,}75\,kg
    \item \textbf{Gewicht 2 (ca.\ 50\,\% MVC):} 10{,}00\,kg
    \item \textbf{Gewicht 3 (ca.\ 75\,\% MVC):} 15{,}75\,kg
\end{itemize}

\subsection{Muskelermüdung}

Die Untersuchung der Muskelermüdung wurde ebenfalls von einem Gruppenmitglied durchgeführt. 
Der Teilnehmer nahm dieselbe Sitzposition wie bei der MVC-Messung ein. 
Es wurden isometrische Kontraktionen von jeweils 15 Sekunden durchgeführt, gefolgt von Pausen von 60 Sekunden. 
Dieser Ablauf wurde dreimal wiederholt.

Innerhalb jedes Durchlaufs wurden drei Zeitpunkte ausgewählt – zu Beginn, in der Mitte und am Ende der Kontraktion – um Veränderungen 
im Frequenzspektrum und Verschiebungen der Medianfrequenz zu analysieren. 
Die Rohdaten wurden in CSV-Dateien gespeichert und anschließend in Python gefiltert, gleichgerichtet und in Einhüllenden umgerechnet. 
Auf Grundlage dieser Daten konnten die Muskelaktivität, die spektrale Leistungsdichte sowie die Medianfrequenz über die Dauer der Ermüdung 
berechnet werden.

\subsection{Zusammenfassung der Durchführung}

Die Durchführung umfasste:
\begin{itemize}
    \item Aufbau und Dokumentation des Messsystems (Arduino, ADC, EMG-Sensor, Elektroden)
    \item Vorbereitung der Elektroden und Sicherstellung einer hohen Signalqualität
    \item Aufnahme der Rohdaten über Python und Speicherung in CSV-Dateien
    \item Durchführung der drei Experimente:
    \begin{itemize}
        \item MVC-Messung für alle Gruppenmitglieder
        \item Messung der relativen Muskelaktivität für ein Gruppenmitglied
        \item Untersuchung der Muskelermüdung für ein Gruppenmitglied
    \end{itemize}
    \item Gewährleistung ausreichender Pausen vor und nach den Kontraktionen zur Minimierung von Artefakten
    \item Bereitstellung der Rohdaten für die spätere Signalverarbeitung (Filterung, Gleichrichtung, Einhüllende, Frequenzanalyse, Berechnung der MVC und relativen Muskelaktivität)
\end{itemize}