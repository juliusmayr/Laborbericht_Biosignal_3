\subsection{Vorverarbeitung der EMG-Daten}

Die aufgezeichneten EMG-Rohdaten wurden vor der eigentlichen Analyse vorverarbeitet, um das physiologisch relevante Muskelaktivitätssignal
von Störanteilen zu trennen und eine robuste Bestimmung der maximalen Muskelaktivität zu ermöglichen.
Zunächst wurde der Mittelwert des EMG-Signals entfernt, um einen Offset der Nulllinie zu korrigieren, der durch das Messsystem 
oder die Elektroden entstehen kann. Anschließend erfolgte eine Bandpassfilterung mit einem Butterworth-Filter im Frequenzbereich von 
20 bis 450\,Hz. Dieser Frequenzbereich entspricht dem typischen Spektrum von Oberflächen-EMG-Signalen und reduziert sowohl 
niederfrequente Bewegungsartefakte als auch hochfrequentes elektrisches Rauschen. 

Im nächsten Schritt wurde das gefilterte Signal gleichgerichtet, indem der Absolutwert gebildet wurde. 
Dadurch wird die wechselnde Polarität des EMG-Signals eliminiert und die Muskelaktivität unabhängig vom Vorzeichen des Signals dargestellt. 
Abschließend wurde aus dem gleichgerichteten Signal eine Einhüllende berechnet, indem ein Tiefpassfilter mit einer Grenzfrequenz 
von 3\,Hz angewendet wurde. Diese Einhüllende beschreibt den zeitlichen Verlauf der Muskelaktivität und dient zur Bestimmung der maximalen
Kontraktionsstärke (vgl. Abbildung~\ref{fig:Vorverarbeitung}).

Die einzelnen Schritte der Vorverarbeitung sind notwendig, da:
\begin{itemize}
    \item die Offsetkorrektur eine Verfälschung der Signalamplitude durch eine verschobene Nulllinie verhindert,
    \item die Bandpassfilterung das relevante EMG-Frequenzband isoliert und Störsignale reduziert,
    \item Gleichrichtung und Einhüllendenbildung eine quantitative Beschreibung der Muskelaktivität ermöglichen.
\end{itemize}

\begin{figure}[h]
    \centering
    \includegraphics[width=0.9\textwidth]{Laborbericht_Biosignal_3/Graphen/Vorverarbeitung.png}
    \caption{Vorverarbeitung der EMG-Daten während der MVC-Messung. (a) Bandpassgefiltertes EMG-Signal nach Offset-Korrektur, (b) gleichgerichtetes EMG-Signal, (c) Einhüllende des EMG-Signals nach Tiefpassfilterung (3 Hz).}
    \label{fig:Vorverarbeitung}
\end{figure}
