% sections/einleitung.tex

In diesem Laborversuch werden biosignalbasierte Messdaten aufgenommen, vorverarbeitet und analysiert.
Ziel ist es, den vollständigen Workflow von der Datenerfassung über die Signalaufbereitung bis zur
quantitativen Auswertung nachvollziehbar zu dokumentieren.

Im Fokus stehen dabei (i) die Auswahl geeigneter Messparameter und Versuchsbedingungen, (ii) die
Qualitätssicherung der Rohdaten (z.\,B. Erkennen von Artefakten und Störungen) sowie (iii) die
Ableitung relevanter Kennwerte aus dem Signal. Die Auswertung erfolgt anhand klar definierter
Kriterien, sodass die Ergebnisse reproduzierbar sind.

Der Bericht ist wie folgt aufgebaut: Zunächst werden die theoretischen Grundlagen und die für die
Interpretation relevanten Signalcharakteristika beschrieben. Anschließend werden Versuchsaufbau und
Messdurchführung dargestellt. Darauf aufbauend folgen die Auswertemethodik und die Ergebnisse.
Abschließend werden die Resultate diskutiert und in einem Fazit zusammengefasst.
