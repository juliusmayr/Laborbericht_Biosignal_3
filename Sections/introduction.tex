Ziel dieser Laboreinheit war der Aufbau, die Inbetriebnahme und die Anwendung 
eines Elektromyografie-(EMG-)Messsystems zur Erfassung und Analyse muskulärer Aktivität. 
Die Studierenden sollten ein grundlegendes Verständnis dafür entwickeln, wie elektrische Signale der Skelettmuskulatur mit Hilfe 
von Oberflächenelektroden erfasst, verstärkt, digitalisiert und rechnergestützt ausgewertet werden können. 
Ein besonderer Fokus lag auf der Untersuchung des Zusammenhangs zwischen Muskelaktivierung, äußerer Belastung und Ermüdung.
Neben der praktischen Durchführung der Messungen wurde die Qualität der EMG-Signale analysiert und der Einfluss typischer Störquellen 
wie Bewegungsartefakte und Kabelbewegungen. Durch geeignete Vorverarbeitungsschritte wie Offsetkorrektur, Bandpassfilterung, 
Gleichrichtung und Hüllkurvenbildung sollten aussagekräftige Kenngrößen der Muskelaktivität gewonnen werden. 
Darüber hinaus wurde eine Frequenzanalyse durchgeführt, um Veränderungen der spektralen Eigenschaften des EMG-Signals im Verlauf einer 
Muskelermüdung zu untersuchen und diese mit der Rekrutierung unterschiedlicher Muskelfasertypen in Beziehung zu setzen.


\section{Aufgaben}

Die Hauptaufgaben der Übung umfassten:
\begin{itemize}[leftmargin=*] % * entfernt Einrückung
    \item Aufbau und Inbetriebnahme des EMG-Messsystems
    \item Platzierung von Oberflächenelektroden am \textit{Musculus biceps brachii}
    \item Aufnahme von EMG-Rohdaten bei maximaler willkürlicher Kontraktion (MVC), relativer Belastung und Ermüdung
    \item Anwendung von Vorverarbeitungsschritten auf die EMG-Rohdaten (Offsetkorrektur, Filterung, Gleichrichtung, Hüllkurvenbildung)
    \item Berechnung der MVC sowie der relativen Muskelaktivität in Prozent der MVC
    \item Durchführung einer Frequenzanalyse und Berechnung der Medianfrequenz zur Bewertung der Muskelermüdung
\end{itemize}

\section{Verwendete Hardware}

Das EMG-Messsystem besteht aus mehreren Hardware- und Softwarekomponenten, die über definierte Signalpfade miteinander verbunden sind (siehe Abbildung~\ref{fig:setup}). Jede Komponente erfüllt eine spezifische Funktion in der Erfassung, Digitalisierung und Übertragung der EMG-Signale.  
Beschreibung der Komponenten und Signalpfade:

\begin{itemize}[leftmargin=*]
    \item \textbf{Mikrocontroller:} \\
    Der Mikrocontroller dient als zentrale Verarbeitungseinheit des Systems. Er steuert die Kommunikation mit dem externen Analog-Digital-Konverter und überträgt die digitalisierten Messdaten über die serielle USB-Schnittstelle an den Computer.

    \item \textbf{EMG/EKG-Sensor:} \\
    Der EMG-Sensor erfasst die elektrischen Potentiale der Muskulatur über Oberflächenelektroden, verstärkt das Signal und stellt es als analoges Ausgangssignal zur weiteren Digitalisierung bereit.

    \item \textbf{Oberflächenelektroden:} \\
    Zwei Messelektroden werden entlang des Muskelbauchs des \textit{Musculus biceps brachii} platziert, während eine Referenzelektrode an einem knöchernen, bewegungsarmen Bereich angebracht wird. Sie bilden die Schnittstelle zwischen physiologischem Signal und Messsystem.

    \item \textbf{12-Bit Analog-Digital-Konverter (ADC):} \\
    Der externe ADC digitalisiert das analoge EMG-Signal mit höherer Auflösung als der interne ADC des Mikrocontrollers. Die Kommunikation erfolgt über den I\textsuperscript{2}C-Bus mittels Qwiic-Kabel.

    \item \textbf{Jumper- und Qwiic-Kabel:} \\
    Jumper-Kabel dienen der Übertragung von Versorgungsspannung und analogem Signal zwischen Sensor und ADC. Das Qwiic-Kabel stellt die digitale I\textsuperscript{2}C-Verbindung zwischen ADC und Mikrocontroller her.
\end{itemize}

\begin{figure}[h]
    \centering
    \includegraphics[width=0.8\textwidth]{bild.png} % Hier Bilddatei einsetzen
    \caption{Setup des EMG-Messsystems}
    \label{fig:setup}
\end{figure}

Dieses Setup ermöglicht eine hochauflösende Erfassung von EMG-Signalen und stellt sicher, dass die Messdaten für die zeit- und frequenzbasierte Analyse der Muskelaktivität und Ermüdung geeignet sind. Durch die klare Zuordnung der Signalpfade und Kommunikationsschnittstellen wird die Funktionsweise des Systems transparent und nachvollziehbar dargestellt.

\section{Rahmenbedingungen}

Die Laborsitzung fand am 15.12.2025 im Medizintechniklabor des Management Center
Innsbruck (MCI) statt. Die Durchführung erfolgte von Jakob Haas, Julius Mayr und Sophia
Nowotny unter der Anleitung von Aitor Morillo und Gerda Strutzenberger.

